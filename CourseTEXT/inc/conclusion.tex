\anonsection{ЗАКЛЮЧЕНИЕ}
\begin{enumerate} 
	\item Проведен обзор существующего программного обеспечения.
Рассмотрены приложения Unison, GoodSync, FreeFileSync, FolderSync и Syncthing.
Наиболее интересной является Syncthing, так как она обеспечивает безопасность, доступность и конфиденциальность данных и имеет открытый исходный код.

	\item Изучен криптографический протокол TLS и рассмотрен процесс установления рукопожатий для двух его актуальных версий: TLS 1.2 и TLS 1.3. Также рассмотрен процесс создания самозаверенных сертификатов, на основе которых происходит аутентификация участника.

	\item Для разработки клиентского приложения выбран язык программирования Python, поскольку он является кросс-платформенным, обладает динамической типизацией и очисткой памяти, а также благодаря наличию фреймворка kivy, который позволяет создавать мобильные приложения с естественным пользовательским интерфейсом.

В качестве среды разработки для всего проекта был использован Visual Studio Code, так как он позволяет выполнять быструю навигацию по коду, автоматически дополняет типовые конструкции и выводит контекстные подсказки.

Серверная часть ПО написана на языке Си, так как он является переносимым и одним из самых быстрых высокоуровневых языков программирования.

Для сборки использовался компилятор GCC, отладка происходила при помощи отладчика LLDB, а для проверки на наличие утечек памяти использовался  Valgrind.

	\item Разработано программное обеспечение для защищенной синхронизации данных между мобильным устройством и персональным компьютером со следующими возможностями:

 1) однонаправленная синхронизация данных с телефона на ПК;

 2) однонаправленная синхронизация данных с ПК на телефон;

 3) двунаправленная синхронизация;

 4) удаленное завершение работы сервера;
 
 5) отслеживание конфликтов при синхронизации;

\end{enumerate}