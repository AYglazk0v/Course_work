\anonsection{ВВЕДЕНИЕ}

В современном мире человек окружен информационными устройствами, основными в его эксплуатации являются смартфон и персональный компьютер. А так как современный темп жизни требует от людей больше мобильности, то всё чаще работать с данными приходится на разных устройства, при этом сохраняя доступ к актуальной версии используемого файла. Для осуществления этого существует синхронизация данных.

Синхронизация данных — процедура копирования или удаления файлов с целью приведения двух каталогов к идентичному состоянию. Предполагается, что ранее эти копии были одинаковы, а затем одна из них, либо обе были независимо изменены.

Чтобы получить актуальную версию файла на рабочем устройстве, прежде всего необходим канал передачи данных. Исходя из необходимости сохранять мобильность, удобно использовать Wi-Fi.

Однако на такой способ передачи данных может быть проведена атака посредника – ``человек посередине'' – когда злоумышленник ретранслирует информацию через себя и при необходимости может её изменить. Исходя из этого, также необходимо обеспечить безопасность канала передачи.

Целью данной курсовой работы является разработка программного обеспечения для защищенной синхронизации данных между мобильным устройством и ПК.

Поставленная цель потребовала решения следующих задач:

\begin{enumerate}
\item Провести обзор существующего ПО для синхронизация данных между мобильным устройством и ПК.
\item Изучить протокол защищенной передачи данных TLS.
\item Выбрать язык программирования и изучить соответствующие средства разработки для ОС Android.
\item Разработать серверное приложение защищенной синхронизации данных.
\item Разработать клиентское приложение защищенной синхронизации данных для ОС Android.
\end{enumerate}