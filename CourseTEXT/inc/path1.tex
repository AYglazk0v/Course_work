\section{\nohyphens{ОБЗОР СУЩЕСТВУЮЩЕГО ПО ДЛЯ СИНХРОНИЗАЦИИ ДАННЫХ МЕЖДУ МОБИЛЬНЫМ УСТРОЙСТВОМ И ПК}}

\subsection{Программное обеспечение Unison}
Unison --- это инструмент для синхронизации файлов в  POSIX-совместимых системах, Windows и macOS\cite{unison}. Он поддерживает два типа интерфейса: GUI и CLI.\\
К достоинствам программы можно отнести:
\begin{enumerate}
	\item Кроссплатформенность.
	\item Возможность работы с ``конфликтами''.
	\item Возможность синхронизации по SSH.
	\item Открытый исходный код.
\end{enumerate}
К недостаткам можно отнести — отсутствие клиента под Android, для синхронизации через сеть потребуется эмулятор терминала.

\subsection{Программное обеспечение GoodSync}

GoodSync --- программа, осуществляющая функции резервного копирования и синхронизации файлов, поддерживает большинство ОС: Windows, Android, iOS и OS X\cite{goodsync}. Синхронизация может осуществляться в каталогах одного компьютера, либо же между компьютером и другим устройством хранения информации, в том числе удаленным сервером.
Достоинства:
\begin{enumerate}
	\item Гибкая настройка масок, позволяющая выбрать, какие именно папки и файлы необходимо синхронизировать.
	\item Обнаружение ``конфликтов'', в случае, когда файлы были изменены в сразу в двух источниках.
	\item Возможность отслеживать факт удаления файлов.
	\item Исключение повторного копирования, программа может определить, когда было изменено только время или состояние файла, и изменить его на другой стороне, без копирования файла.
	\item Синхронизация файлов с облачным хранилищем.
	\item Возможность запланировать синхронизацию с определенным интервалом времени.
\end{enumerate}
Недостатки:
\begin{enumerate}
	\item Программа с закрытым исходным кодом.
	\item Необходима регистрация на сайте проекта.
	\item GoodSync является условно бесплатным: большинство функций работают только в платной версии.
\end{enumerate}

\subsection{Программное обеспечение FreeFileSync}
FreeFileSync --- это бесплатная программа с открытым исходным кодом, используемая для синхронизации и создания резервных копий\cite{freefilesync}. Синхронизация обеспечивается посредством нахождения различий в целевых файлах и передачей только измененной части.
Достоинства:
\begin{enumerate}
	\item Открытый исходный код.
	\item Возможность синхронизации по параметрам: дата модификации и размер, содержание файла, размер файла.
	\item Возможность настройки сценариев синхронизации.
	\item Экономичное использование ресурсов ПК.
	\item Гибкая настройка параметров синхронизации: какие файлы и папки синхронизировать, что игнорировать.
	\item Встроенная поддержка FTP, FTPS, SFTP и MTP.
\end{enumerate}

Из недостатков можно отметить: отсутствие официального клиента для мобильных устройств, синхронизация возможна только по MTP, либо через развертывание FTP сервера на мобильном устройстве.

\subsection{Приложение FolderSync}
FolderSync --- это Android приложение, которое обеспечивает простую синхронизацию с облачными хранилищами, либо же в локальных папках устройства\cite{foldersync}.
К достоинствам приложения можно отнести:
\begin{enumerate}
	\item Гибкая настройка фильтров синхронизации.
	\item Возможность запланировать интервалы синхронизации файлов.
	\item Большой перечень поддерживаемых облачных хранилищ для синхронизации.
	\item Поддержка протоколов: FTP, FTPES, FTPS, SFTP, WebDAV, SMB1 и SMB2, что позволяет обеспечить безопасную передачу данных с устройства на сервер.
\end{enumerate}
Недостатки:
\begin{enumerate}
	\item Программа с закрытым исходным кодом.
	\item Условно бесплатное приложение: бесплатная версия имеет ограниченный функционал и содержит рекламу.
	\item Большой перечень требуемых разрешений: ACCESS\_FINE\_LOCATION, ACCESS\_NETWORK\_STATE, ACCESS\_WIFI\_STATE, WAKE\_LOCK и др.
\end{enumerate}

\subsection{Программное обеспечение Syncthing}
Syncthing — это бесплатное кроссплатформенное приложение открытым исходным кодом для непрерывной синхронизации файлов\cite{syncthing}. Оно может синхронизировать файлы между устройством в локальной сети и между удаленными устройствами через интернет с условием обеспечения защиты и безопасности данных. Данное приложение позволяет создать свое облачное хранилище.
К достоинствам этого приложения можно отнести:
\begin{enumerate}
	\item Приложение с открытым исходным кодом.
	\item Версионность.
	\item Кроссплатформенность.
	\item Открытый протокол, на основе которого работает приложение.
	\item Экономичное использование ресурсов ПК.
	\item Отсутствие регистрации.
	\item Гибкая настройка СКД.
	\item Синхронизации на уровне блоков, то есть при маленьком изменении в большом файле будет обновляться только небольшая часть файла.
\end{enumerate}

Из недостатков можно отметить: низкую скорость передачи данных при нахождении устройств в разных сетях.

\subsection{Выводы о представленном ПО}
Существует большое количество разнообразного программного обеспечения для осуществления синхронизации данных. 

Наиболее интересной среди рассмотренных программ является Syncthing, так как она обеспечивает безопасность --- для доступа к файлам синхронизации необходимо подтверждение у владельца данных, при передаче же данных весь трафик шифруется, доступность --- файлы доступны как внутри сети владельца, так и из другой сети, конфиденциальность --- Syncthing выложен на GitHub и не требует никакой личной информации для использования.