\section{ТЕСТИРОВАНИЕ РЕАЛИЗОВАННОГО ПО}
\subsection{Интерфейс разработанного приложения}
После установки приложения на рабочем столе телефона появится ярлык "CoursWork" для запуска нашего проекта (рис. \ref{p1}a).

После запуска оно попросит разрешить доступ к файлам, это необходимо для подгрузки сертификатов и в целом для синхронизации (рис. \ref{p1}б).

Затем приложение уведомит о том, что доступ к сертификатам получен (рис. \ref{p1}в) и пользователю будет предложено ввести IP адрес сервера (рис. \ref{p1}г и рис. \ref{p2}а).

После ввода IP адреса, в случае успешного соединения будет выведено сообщение о том, что TLS рукопожатие прошло успешно(рис. \ref{p2}б). Далее будут доступны 5 кнопок (рис. \ref{p2}в): 
\begin{enumerate}
	\item Синхронизировать ПК --- однонаправленная синхронизация при которой файлы на ПК будут обновлены в случае, если версия на мобильном телефоне новее, либо же созданы, если на ПК они отсутствовали.
	\item Синхронизировать телефон --- однонаправленная синхронизация аналогичная предыдущей, но обновления будут произведены на телефоне.
	\item Синхронизировать оба устройства --- двунаправленная синхронизация при которой будут обновлены файлы на ПК и телефоне.
	\item Выход --- осуществление отключение от сервера и выхода из приложения.
	\item Выход и отключение сервера --- аналогично предыдущему, но в этом случае свою работу также завершит сервер.
\end{enumerate}

После выбора метода синхронизации откроется файловый менеджер(рис. \ref{p2}г), позволяющий выбрать каталог для синхронизации. Для подтверждения каталога необходимо нажать на галочку.

После успешной синхронизации на экране появится сообщение о том, что синхронизация была выполнена (рис.\ref{p3}).
\addtwoxtwoimghere{pix/p1}{pix/p2}{pix/p3}{pix/p4}{Интерфейс приложения при запуске}{p1}
\addtwoxtwoimghere{pix/p5}{pix/p6}{pix/interf}{pix/p7}{Интерфейс приложения}{p2}
\addimghere{pix/p8}{0.35}{Завершение синхронизации}{p3}



\subsection{Синхронизация ПК}
Проведем синхронизацию данных с телефона на ПК, изначально каталог \~{}/CourseWorkProj/ пуст. 

На телефоне создадим директорию TestSync и добавим файлов, разных форматов и стилем наименования и объемом данных (рис. \ref{p12})
\addimghere{pix/p12}{0.35}{Содержимое каталога для тестирования}{p12}
В приложении для синхронизации выберем режим ``Синхронизация ПК'', а затем подготовленный каталог ``TestSync'', после этого в терминале серверного приложения начнется вывод информации о производимых действиях (рис. \ref{s2} и рис. \ref{s2s})
\addimghere{pix/s2}{0.7}{Вывод терминала во время синхронизации}{s2}
\addimghere{pix/s3}{0.7}{Вывод терминала во время синхронизации (продолжение)}{s2s}

После завершения синхронизации проверим содержимое \~{}/CourseWorkProj/TestSync (рис. \ref{s4}). Количество файлов, их объем и временные метки совпадают с теми, что были на телефоне, соответственно синхронизация прошла успешно.
\addimghere{pix/s4}{1}{Вывод терминала во время синхронизации}{s4}

\subsection{Синхронизация телефона}
В каталог \~{}/CourseWorkProj/TestSync добавим файлы (рис. \ref{newfile}), на мобильном телефоне выберем режим ``Синхронизация телефона'', а затем каталог ``TestSync'', после этого в терминале серверного приложения начнется вывод информации о производимых действиях (рис. \ref{newlog} и рис. \ref{newlog2}). Просмотрим содержимое каталога TestSync на телефоне (рис. \ref{sytel} а - г).
\addimghere{pix/newfile}{0.8}{Содержимое синхронизируемого каталога}{newfile}
\addimghere{pix/newlog}{0.8}{Вывод терминала во время синхронизации}{newlog}
\addimghere{pix/newlog2}{0.8}{Вывод терминала во время синхронизации (продолжение)}{newlog2}
\addtwoxtwoimghere{pix/p14}{pix/p16}{pix/p17}{pix/p18}{Содержимое TestSync на телефоне после синхронизации}{sytel}

\newpage
\subsection{Двусторонняя синхронизация}
Для демонстрации двусторонней синхронизации модифицируем файлы:

На ПК --- удалим символ файле war\_and\_pice\_3.txt, а в 55.jpg изменим контрасность.

На телефоне --- поднимем экспозицию у файлов 5.jpg и 9.jpg.

После внесенных изменений зайдем в наше приложение, выберем и двустороннюю синхронизацию,  каталог ``TestSync'' и пронаблюдаем за происходящим в терминале серверной части проекта (рис. \ref{towaysync}). По его содержимому видно, что передавались только измененные файлы.

\addimghere{pix/towaysync}{0.7}{Вывод терминала во время синхронизации}{towaysync}
\newpage
\subsection{Попытка подключиться без сертификата}
Посмотрим на то, как отреагирует сервер, если мы попытаемся подключиться к нему без сертификатов, либо же с сертификатами, подписанными другим доверенным центром.

Для этого в директории с сертификатами заменим наши сертификаты, на недавно сгенерированные, а перед этим просто удалим и попытаемся подключиться к серверу.

Как видно из терминала серверного приложения (рис. \ref{s12}), при попытке подключения без сертификата, либо же с другим, сервер обрывает соединение и ждет нового пользователя.

\addimghere{pix/s12}{0.6}{Вывод терминала во время попытки подключения}{s12}

\subsection{Попытка передачи пустого файла}
В связи с особенностями реализации некоторых библиотечных функций Python, возможен сбой в работе алгоритма при попытке передачи пустого файла. Проверим, как отрабатывает наше ПО.

Для этого на ПК создадим пустой файл pc\_empty, на телефоне, соответственно, phone\_empty.

Запустим двустороннюю синхронизацию и в результате получим, что оба файла передались между устройствами: на ПК (рис. \ref{ef}) и на телефон (рис. \ref{ef1}).

\addimghere{pix/e1}{1}{Проверка наличия пустых файлов на ПК после синхронизации}{ef}

\addimghere{pix/e2}{0.35}{Проверка наличия пустых файлов на телефоне после синхронизации}{ef1}

\subsection{Попытка синхронизировать файл и папку с одинаковыми названиями}
Создадим на ПК каталог с названием ``ErrFile'', а на телефоне файл с таким же именем. Попробуем выполнить синхронизацию. Так как в тестовом каталоге больше нет файлов, которые были изменены, то в терминале мы увидим единственную запись -- конфликт (рис. \ref{error1}). Так же на экране мобильного устройства появится название файла в котором произошел конфликт (рис. \ref{error2}).

\addimghere{pix/er16}{0.8}{Вывод терминала во время синхронизации}{error1}

\addimghere{pix/e}{0.35}{Экран мобильного устройства при наличии конфликтных файлов}{error2}
\subsection{Проверка на утечки памяти после завершения синхронизации}
Так как сервер написан на Си, а он не имеет механизмов автоматической очистки памяти, то нужно убедиться, что вся аллоцированая память была освобождена, после выполнения синхронизации. Сначала добавим новых файлов в директорию синхронизации на телефоне и ПК. Далее запустим сервер следующим образом:
\begin{minted}[]{C}
	valgrind --leak-check=full -s ./coursework
\end{minted} 
Результат выполнения представлен на рис. \ref{leak1} и рис. \ref{leak2}. 
\addimghere{pix/l2}{0.7}{Старт проверки на наличие утечек}{leak1}
\addimghere{pix/l1}{0.7}{Конец проверки на наличие утечек}{leak2}

