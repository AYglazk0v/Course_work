\anonsection{ИСПОЛЬЗУЕМЫЕ ТЕРМИНЫ И СОКРАЩЕНИЯ}
ОС --- операционная система

ПК --- персональный компьютер

СКД --- система контроля и управления доступом

Клиент --- участник клиент-серверной архитектуры, инициирующий 
соединение с сервером.

Сервер --- участник клиент-серверной архитектуры, принимающий входящее соединение от клиента.

Хэш (англ. Hash) – набор символов фиксированной длины, сгенерированный на основе входных данных произвольной длины и однозначно идентифицирующий их.

Синхронизация данных --- процедура копирования или удаления файлов с целью приведения двух каталогов к идентичному состоянию.

Конфликт синхронизации данных --- ситуация, возникающая при синхронизации информации, когда невозможно синхронизировать файл или каталог.

Поддержка обновлений --- возможность синхронизировать только измененный фрагмент файла.

Версионность --- возможность оперировать с версиями синхронизируемых файлов.

Интерфейс командной строки (CLI) --- текстовый пользовательский интерфейс (UI), который используется для запуска программ, работы с компьютерными файлами и взаимодействия с компьютером. 

Графический интерфейс пользователя (GUI) --- форма пользовательского интерфейса, которая позволяет пользователям взаимодействовать с электронными устройствами посредством графических значков и звуковых индикаторов, вместо текстовых пользовательских интерфейсов, вводимых командных меток или текстовой навигации.

ПО с открытым исходным кодом --- вид ПО, исходный код которого доступен для просмотра и изменения. Это позволяет использовать уже созданный код для создания новых версий программ, дает возможность для исправления ошибок, а также позволяет убедиться в отсутствии недекларированных функций.

ПО с закрытым исходным кодом --- вид ПО, все права на использование, изменение и копирование которого принадлежат его автору. Одним из его недостатков является возможность наличия недекларированных функций.

